\documentclass[../ESOF_notes.tex]{subfiles}

\begin{document}

\subsection{Part I – Software Reviews \& Inspections}
\subsubsection{Verification versus Validation}
\begin{itemize}
    \item \textbf{Verification – are we building the product right}
          \begin{itemize}
              \item Ensure (mainly through reviews) that intermediate
                    work products and the final product are “well built”, 
                    i.e., conform to their specifications.
          \end{itemize}
    \item \textbf{Validation – are we building the right product?}
          \begin{itemize}
              \item Ensure (manly through tests) that the final
                    product will fulfill its intended use in its intended 
                    environment.
              \item Can also be applied to intermediate work products,
                    as predictors of how well the final product will satisfy 
                    user needs.
          \end{itemize}
\end{itemize}
Verification shows conformance with specification.
Validation shows that the program meets the customer’s needs.
\begin{figure}[h]
    \centering
    \includegraphics[width=12cm]{Validation_Verification.png}
\end{figure}

\subsubsection{Static and Dynamic V\&V Techniques}
\begin{itemize}
    \item \textbf{Static Techniques} –– involve analyzing the
          static system representations to find problems and 
          evaluate quality.
          \begin{itemize}
              \item Reviews and inspections.
              \item Automated static analysis (e.g., with lint).
              \item Formal verification (e.g., with Dafny)
          \end{itemize}
    \item \textbf{Dynamic Techniques} – involve executing the system and
          observing its behavior.
          \begin{itemize}
              \item Software testing.
              \item Simulation.
          \end{itemize}
\end{itemize}

Static verification techniques involve examination and
analysis of the program for error detection. Dynamic
techniques involve executing the program for error
detection. 

They are complementary and not opposing techniques.
Both should be used during the V\&V process.
\pagebreak       
\subsubsection{Software reviews and inspections}

Analysis of static system representations to find problems.
\begin{itemize}
    \item Manual analysis of requirements specs,
          design specs, code, etc.
    \item  May be supplemented by tool-based
          static analysis.
\end{itemize}

Advantages (as compared to testing): 
\begin{itemize}
    \item Can be applied to any artefact, and not only code
    \item Can be applied earlier (thus reducing impact and cost of errors)
    \item Fault localization (debugging) is immediate
    \item Allows evaluating internal quality attributes (e.g., maintainability)
    \item Usually more efficient and effective than testing in finding security
          vulnerabilities and checking exception handling
    \item Very effective in finding multiple defects
    \item Peer reviews promote knowledge sharing
\end{itemize}

\subsubsection{Efficiency of defect removal methods}
\begin{figure}[h!]
    \centering
    \includegraphics[width=12cm]{minutes_per_defect.png}
\end{figure}
\pagebreak
\subsubsection{Types of Reviews}
\begin{figure}[h!]
    \centering
    \includegraphics[width=12cm]{type_review.png}
\end{figure}

\subsubsection{Review Best Practices}
\begin{itemize}
    \item \textbf{Use a checklist derived from historical defect data:}
          \begin{itemize}
              \item Makes the review more effective and efficient, by focusing
                    the attention on the most frequent and important problems. 
              \item CPersonal checklists make sense, because each person
                    tends to repeat his/her own mistakes.
                    
          \end{itemize}
    \item \textbf{Take enough review time :} 200 LOC/hour
          is a recommended review rate by some authors(LOC-Lines of Code).
    \item \textbf{Take a break between developing and reviewing
              (in personal reviews).}
    \item \textbf{Combine personal reviews with peer reviews or team
              inspections:} Team inspections comprise individual reviews 
          performed by 2+ peers (Individual preparation), followed 
          by a meeting (Inspection meeting) with the producer and 
          possibly a moderator.
          \begin{figure}[h!]
              \includegraphics[width=12cm]{inspection_process.png}
              \centering
              \includegraphics[width=12cm]{defects_found.png}
          \end{figure}
          \pagebreak
    \item \textbf{Measure the review process \& use data to improve:}
          size, time spent, defects found, defects escaped (found later).
\end{itemize}
\pagebreak
\subsubsection{Estimate Missed defects}

The capture-recapture method is used to estimate the total
defects (T) and number of defects remaining (R) based on
the degree of overlapping between defects detected by
different inspectors (A, B).
\begin{figure}[h!]
    \centering
    \includegraphics[width=12cm]{capture_recapture.png}
\end{figure}

In case of more than 2 inspectors, \textbf{A} refers to the inspector that found
more unique defects, and \textbf{B} refers to the union of all other inspectors


\subsection{Part II – Software Testing}
\subsubsection{Test Concepts}
\paragraph{Testing goals and limitations}

Goals:
\begin{itemize}
    \item exercise the software with defined test cases
          and observe its behaviour to discover defects.
    \item  increase the confidence on the software
          correctness and to evaluate product quality.
\end{itemize}

Limitations:
\begin{itemize}
    \item Testing can show the presence of bugs, not their absence
\end{itemize}
\paragraph{Test Cases}
\begin{itemize}
    \item \textbf{Test Case:} A set of test inputs,
          execution conditions, and expected results developed
          to exercise a particular program path.
    \item \textbf{Test Script:} concrete definition of test
          steps / procedure( can be parameterized for reuse with 
          multiple test data).
\end{itemize}
\paragraph{Test Activities}
\begin{itemize}
    \item \textbf{Test Planning:} define the objectives of
          testing and the approach for meeting test objectives 
          within constraints imposed by the context.
    \item \textbf{Test monitoring and control:} compare actual
          progress against the plan, and take actions necessary 
          to meet the objectives of the test plan.
    \item \textbf{Test analysis:} identify testable features
          and test conditions.
    \item \textbf{Test design:} derive test cases.
    \item \textbf{Test implementation:} create automated scripts.
    \item \textbf{Test execution:} run test suites.
    \item \textbf{Test completion:} collect data from completed
          test activities.
\end{itemize}
\paragraph{Test Types}
\begin{figure}[h!]
    \centering
    \includegraphics[width=12cm]{test_types.png}
\end{figure}
\subsubsection{Test Levels}
\paragraph{Unit Testing/Component Testing/Module Testing}
\begin{itemize}
    \item Testing of individual hardware or software units or
          groups of related units.
    \item Detect functional (e.g., wrong calculations)
          and non-functional (e.g., memory leaks) defects in 
          the unit.
    \item Usually API testing.
    \item Responsibility of the developer.
    \item Usually based on experience, specs and code.
\end{itemize}
\paragraph{Integration Testing}
\begin{itemize}
    \item Software and/or hardware components are
          combined and tested to evaluate the interaction
          between them.
    \item Two levels of integration testing:
          \begin{itemize}
              \item \textbf{Component integration testing:}
                    interactions between components.
              \item \textbf{System integration testing:}
                    interactions between systems.
          \end{itemize}
    \item Responsibility of an independent test team.
    \item Usually based on a system spec (technical/design spec).
    \item Detect defects that occur on the units’ interfaces.
    \item For easier fault localisation, integrate
          incrementally/continuously.
\end{itemize}
\paragraph{System Testing}
\begin{itemize}
    \item Conducted on a complete, integrated system
          to evaluate the system's compliance with specified 
          requirements.                
    \item Both functional behavior and quality requirements
          (performance, usability, reliability, security, etc.) 
          are evaluated.   
    \item Usually GUI testing.
    \item Responsibility of an independent test team.
    \item Usually based on requirements document.
\end{itemize}
\paragraph{Acceptance Testing}
\begin{itemize}
    \item Determine whether or not a system atisfies its
          acceptance criteria.
    \item Enable a customer,user, or other authorized entity
          to determine whether or not to accept the system.
    \item Usually the responsibility of the customer.
    \item Based on a requirements document or contract.
    \item Check if customer requirements and expectations
          are met.
\end{itemize}
\paragraph{Regression Testing}
\begin{itemize}
    \item Tests to verify that modifications have not
          caused unintended effects and that the system or 
          component still complies with its specified requirements.
    \item Changes to software, to enhance it or fix bugs,
          are a very common source of defects.
    \item Not really a new test level, but just the repetition
          of testing at any level.
\end{itemize}
\subsubsection{Test case design techniques}
\paragraph{Design goals:}
\begin{itemize}
    \item Create a set of test cases (test suite) that are
          effective in validation and defect testing.
    \item A good test suite should have a small/manageable
          size and have a high probability of finding most of the 
          defects.
\end{itemize}
\paragraph{Design Strategies:}

\textbf{Black-Box Testing:} Derivation of test 
cases based on some external specification.
\begin{itemize}
    \item \textbf{Equivalence class partitioning:}
          partition the input domain into classes of equivalent 
          behavior, separating classes of valid and invalid 
          inputs, and select at least one test case from each
          class.
    \item \textbf{Boundary value analysis:} select test
          values at the boundaries of each partition 
          (e.g., immediately below and above), besides typical 
          values.
    \item \textbf{Decision table testing:} test all possible
          combinations of a set of conditions and actions 
          (each combination corresponding to a business rule). 
    \item \textbf{State transition testing:} derive test
          cases from a state-machine model of the system.
    \item \textbf{Use case testing:} derive test cases
          from a use case model of the system (with use cases 
          possibly detailed with scenarios, pre/post-conditions, 
          etc).
\end{itemize}

\textbf{White-box Testing:} Derivation of test 
cases according to program structure.
\begin{itemize}
    \item \textbf{Using coverage analysis tools:} (e.g., Eclemma)
          to analyse code coverage achieved by black-box tests
          and design additional tests as needed.
    \item  \textbf{Testing statement coverage:} Assure that
          all statements are exercised.
    \item \textbf{Decision/Branch coverage:} Assure that all decisions
          (if, while, for, etc.) take both values true and false. 
\end{itemize}
\subsubsection{Test automation tools}
\begin{itemize}
    \item \textbf{Unit testing frameworks:} JUnit, NUnit.
    \item \textbf{Mock object frameworks:}
          \begin{itemize}
              \item Facilitate simulating external components in unit testing.
              \item EasyMock, jMock.
          \end{itemize}
    \item \textbf{Test coverage analysis tools:}
          \begin{itemize}
              \item Measure degree of code coverage
                    achieved by the execution of a test suite.
              \item Useful for white-box testing.
              \item Eclemma, Clover.
          \end{itemize}
    \item \textbf{Mutation testing tools:}
          \begin{itemize}
              \item Evaluate the quality of a test suite by
                    determining its ability to ‘kill’ mutants
                    (with common fault types) of the program under test.
              \item pitest, muJava.
          \end{itemize}
    \item \textbf{Acceptance testing frameworks:}
          \begin{itemize}
              \item Allows creating test cases by people
                    without technical knowledge.
              \item Cucumber, JBehave, Fitnesse.
          \end{itemize}
    \item \textbf{Capture/replay tools (aka functional testing tools):}
          \begin{itemize}
              \item Capture user interactions in scripts
                    that can be edited and replayed.
              \item Useful for GUI testing, particularly
                    regression testing.
              \item Selenium, IBM Rational Functional Tester.
          \end{itemize}
    \item \textbf{Performance/load testing tools:}
          \begin{itemize}
              \item Execute test suites simulating many users
                    and measure system performance.
              \item IBM Rational Performance Tester, Compuware QA Load.
          \end{itemize}
    \item \textbf{Penetration testing tools:} Metasploit, ZAP.
    \item \textbf{Test case generation tools:}
          \begin{itemize}
              \item Automatically generate test cases from
                    models or code.
              \item IParTeG (UML), EvoSuite (Java), Spec Explorer (Spec\#),
                    Conformiq (UML).
          \end{itemize}
\end{itemize}
\subsubsection{Test management}

Tools:
\begin{itemize}
    \item Manage test information and status.
    \item Integrate with other management tools:
          requirements management, project management, 
          bug tracking, configuration management.
    \item Integrate with test automation tools.
    \item TestLink.
\end{itemize}

Test Management charts are used to track progress 
of testing and bug fixing activities.

\subsubsection{Testing best practices}
\begin{itemize}
    \item \textbf{Test as early as possible} The cost of finding
          and fixing bugs increases exponentially with time.
    \item \textbf{Automate the tests}.
    \item \textbf{Test first (write tests before the code):}
          Helps clarifying requirements and specifications since
          test cases are partial specifications of system behavior.
    \item \textbf{Black-box first:} Start by designing test
          cases based on the specification and then add any tests 
          needed to ensure code coverage.
\end{itemize}
\pagebreak
\subsubsection{Useful Kahoots}
\begin{figure}[h!]
    \includegraphics[width=4cm]{ST-11_1.png}
    \centering
    \includegraphics[width=4cm]{ST-11_2.png}
    \includegraphics[width=4cm]{ST-12_2.png}
    \centering
    \includegraphics[width=4cm]{ST-12_1.png}
\end{figure}
\end{document}